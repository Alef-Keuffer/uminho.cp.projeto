\documentclass{article}
\usepackage[a4paper,left=3cm,right=2cm,top=2.5cm,bottom=2.5cm]{geometry}
%% ODER: format ==         = "\mathrel{==}"
%% ODER: format /=         = "\neq "
%
%
\makeatletter
\@ifundefined{lhs2tex.lhs2tex.sty.read}%
  {\@namedef{lhs2tex.lhs2tex.sty.read}{}%
   \newcommand\SkipToFmtEnd{}%
   \newcommand\EndFmtInput{}%
   \long\def\SkipToFmtEnd#1\EndFmtInput{}%
  }\SkipToFmtEnd

\newcommand\ReadOnlyOnce[1]{\@ifundefined{#1}{\@namedef{#1}{}}\SkipToFmtEnd}
\usepackage{amstext}
\usepackage{amssymb}
\usepackage{stmaryrd}
\DeclareFontFamily{OT1}{cmtex}{}
\DeclareFontShape{OT1}{cmtex}{m}{n}
  {<5><6><7><8>cmtex8
   <9>cmtex9
   <10><10.95><12><14.4><17.28><20.74><24.88>cmtex10}{}
\DeclareFontShape{OT1}{cmtex}{m}{it}
  {<-> ssub * cmtt/m/it}{}
\newcommand{\texfamily}{\fontfamily{cmtex}\selectfont}
\DeclareFontShape{OT1}{cmtt}{bx}{n}
  {<5><6><7><8>cmtt8
   <9>cmbtt9
   <10><10.95><12><14.4><17.28><20.74><24.88>cmbtt10}{}
\DeclareFontShape{OT1}{cmtex}{bx}{n}
  {<-> ssub * cmtt/bx/n}{}
\newcommand{\tex}[1]{\text{\texfamily#1}}	% NEU

\newcommand{\Sp}{\hskip.33334em\relax}


\newcommand{\Conid}[1]{\mathit{#1}}
\newcommand{\Varid}[1]{\mathit{#1}}
\newcommand{\anonymous}{\kern0.06em \vbox{\hrule\@width.5em}}
\newcommand{\plus}{\mathbin{+\!\!\!+}}
\newcommand{\bind}{\mathbin{>\!\!\!>\mkern-6.7mu=}}
\newcommand{\rbind}{\mathbin{=\mkern-6.7mu<\!\!\!<}}% suggested by Neil Mitchell
\newcommand{\sequ}{\mathbin{>\!\!\!>}}
\renewcommand{\leq}{\leqslant}
\renewcommand{\geq}{\geqslant}
\usepackage{polytable}

%mathindent has to be defined
\@ifundefined{mathindent}%
  {\newdimen\mathindent\mathindent\leftmargini}%
  {}%

\def\resethooks{%
  \global\let\SaveRestoreHook\empty
  \global\let\ColumnHook\empty}
\newcommand*{\savecolumns}[1][default]%
  {\g@addto@macro\SaveRestoreHook{\savecolumns[#1]}}
\newcommand*{\restorecolumns}[1][default]%
  {\g@addto@macro\SaveRestoreHook{\restorecolumns[#1]}}
\newcommand*{\aligncolumn}[2]%
  {\g@addto@macro\ColumnHook{\column{#1}{#2}}}

\resethooks

\newcommand{\onelinecommentchars}{\quad-{}- }
\newcommand{\commentbeginchars}{\enskip\{-}
\newcommand{\commentendchars}{-\}\enskip}

\newcommand{\visiblecomments}{%
  \let\onelinecomment=\onelinecommentchars
  \let\commentbegin=\commentbeginchars
  \let\commentend=\commentendchars}

\newcommand{\invisiblecomments}{%
  \let\onelinecomment=\empty
  \let\commentbegin=\empty
  \let\commentend=\empty}

\visiblecomments

\newlength{\blanklineskip}
\setlength{\blanklineskip}{0.66084ex}

\newcommand{\hsindent}[1]{\quad}% default is fixed indentation
\let\hspre\empty
\let\hspost\empty
\newcommand{\NB}{\textbf{NB}}
\newcommand{\Todo}[1]{$\langle$\textbf{To do:}~#1$\rangle$}

\EndFmtInput
\makeatother
%
%
%
%
%
%
% This package provides two environments suitable to take the place
% of hscode, called "plainhscode" and "arrayhscode". 
%
% The plain environment surrounds each code block by vertical space,
% and it uses \abovedisplayskip and \belowdisplayskip to get spacing
% similar to formulas. Note that if these dimensions are changed,
% the spacing around displayed math formulas changes as well.
% All code is indented using \leftskip.
%
% Changed 19.08.2004 to reflect changes in colorcode. Should work with
% CodeGroup.sty.
%
\ReadOnlyOnce{polycode.fmt}%
\makeatletter

\newcommand{\hsnewpar}[1]%
  {{\parskip=0pt\parindent=0pt\par\vskip #1\noindent}}

% can be used, for instance, to redefine the code size, by setting the
% command to \small or something alike
\newcommand{\hscodestyle}{}

% The command \sethscode can be used to switch the code formatting
% behaviour by mapping the hscode environment in the subst directive
% to a new LaTeX environment.

\newcommand{\sethscode}[1]%
  {\expandafter\let\expandafter\hscode\csname #1\endcsname
   \expandafter\let\expandafter\endhscode\csname end#1\endcsname}

% "compatibility" mode restores the non-polycode.fmt layout.

\newenvironment{compathscode}%
  {\par\noindent
   \advance\leftskip\mathindent
   \hscodestyle
   \let\\=\@normalcr
   \let\hspre\(\let\hspost\)%
   \pboxed}%
  {\endpboxed\)%
   \par\noindent
   \ignorespacesafterend}

\newcommand{\compaths}{\sethscode{compathscode}}

% "plain" mode is the proposed default.
% It should now work with \centering.
% This required some changes. The old version
% is still available for reference as oldplainhscode.

\newenvironment{plainhscode}%
  {\hsnewpar\abovedisplayskip
   \advance\leftskip\mathindent
   \hscodestyle
   \let\hspre\(\let\hspost\)%
   \pboxed}%
  {\endpboxed%
   \hsnewpar\belowdisplayskip
   \ignorespacesafterend}

\newenvironment{oldplainhscode}%
  {\hsnewpar\abovedisplayskip
   \advance\leftskip\mathindent
   \hscodestyle
   \let\\=\@normalcr
   \(\pboxed}%
  {\endpboxed\)%
   \hsnewpar\belowdisplayskip
   \ignorespacesafterend}

% Here, we make plainhscode the default environment.

\newcommand{\plainhs}{\sethscode{plainhscode}}
\newcommand{\oldplainhs}{\sethscode{oldplainhscode}}
\plainhs

% The arrayhscode is like plain, but makes use of polytable's
% parray environment which disallows page breaks in code blocks.

\newenvironment{arrayhscode}%
  {\hsnewpar\abovedisplayskip
   \advance\leftskip\mathindent
   \hscodestyle
   \let\\=\@normalcr
   \(\parray}%
  {\endparray\)%
   \hsnewpar\belowdisplayskip
   \ignorespacesafterend}

\newcommand{\arrayhs}{\sethscode{arrayhscode}}

% The mathhscode environment also makes use of polytable's parray 
% environment. It is supposed to be used only inside math mode 
% (I used it to typeset the type rules in my thesis).

\newenvironment{mathhscode}%
  {\parray}{\endparray}

\newcommand{\mathhs}{\sethscode{mathhscode}}

% texths is similar to mathhs, but works in text mode.

\newenvironment{texthscode}%
  {\(\parray}{\endparray\)}

\newcommand{\texths}{\sethscode{texthscode}}

% The framed environment places code in a framed box.

\def\codeframewidth{\arrayrulewidth}
\RequirePackage{calc}

\newenvironment{framedhscode}%
  {\parskip=\abovedisplayskip\par\noindent
   \hscodestyle
   \arrayrulewidth=\codeframewidth
   \tabular{@{}|p{\linewidth-2\arraycolsep-2\arrayrulewidth-2pt}|@{}}%
   \hline\framedhslinecorrect\\{-1.5ex}%
   \let\endoflinesave=\\
   \let\\=\@normalcr
   \(\pboxed}%
  {\endpboxed\)%
   \framedhslinecorrect\endoflinesave{.5ex}\hline
   \endtabular
   \parskip=\belowdisplayskip\par\noindent
   \ignorespacesafterend}

\newcommand{\framedhslinecorrect}[2]%
  {#1[#2]}

\newcommand{\framedhs}{\sethscode{framedhscode}}

% The inlinehscode environment is an experimental environment
% that can be used to typeset displayed code inline.

\newenvironment{inlinehscode}%
  {\(\def\column##1##2{}%
   \let\>\undefined\let\<\undefined\let\\\undefined
   \newcommand\>[1][]{}\newcommand\<[1][]{}\newcommand\\[1][]{}%
   \def\fromto##1##2##3{##3}%
   \def\nextline{}}{\) }%

\newcommand{\inlinehs}{\sethscode{inlinehscode}}

% The joincode environment is a separate environment that
% can be used to surround and thereby connect multiple code
% blocks.

\newenvironment{joincode}%
  {\let\orighscode=\hscode
   \let\origendhscode=\endhscode
   \def\endhscode{\def\hscode{\endgroup\def\@currenvir{hscode}\\}\begingroup}
   %\let\SaveRestoreHook=\empty
   %\let\ColumnHook=\empty
   %\let\resethooks=\empty
   \orighscode\def\hscode{\endgroup\def\@currenvir{hscode}}}%
  {\origendhscode
   \global\let\hscode=\orighscode
   \global\let\endhscode=\origendhscode}%

\makeatother
\EndFmtInput
%
\usepackage{stmaryrd}
\usepackage[math-style=ISO]{unicode-math}

%%\defaultfontfeatures{ Scale = MatchUppercase }
%%\setmainfont{TeX Gyre Pagella}[Ligatures = {Common, TeX}, Scale = 1.0]
%%\setmathfont{TeX Gyre Pagella Math}
%%\renewcommand{\texfamily}{\familydefault\selectfont}
%%\renewcommand{\tex}[1]{\text{#1}}











\usepackage{Solucoes}
\usepackage{imakeidx}

\makeindex
\begin{document}
\section{Resolução Alternativa}

\subsection{Questão 1}

Para criar uma interpretação de um tipo \ensuremath{\Conid{A}} como um tipo \ensuremath{\Conid{B}}. Assim, por exemplo,
posso definir que uma expressão $x$ do tipo \ensuremath{\Conid{ExpAr}\;\Varid{a}} pode ser interpretada como
\ensuremath{\Conid{Left}\;()} do tipo \ensuremath{\Conid{OutExpAr}\;\Varid{a}}.
\begin{hscode}\SaveRestoreHook
\column{B}{@{}>{\hspre}l<{\hspost}@{}}%
\column{5}{@{}>{\hspre}l<{\hspost}@{}}%
\column{E}{@{}>{\hspre}l<{\hspost}@{}}%
\>[B]{}\mathbf{class}\;\Conid{Interpretation}\;\Varid{a}\;\Varid{b}\;\mathbf{where}{}\<[E]%
\\
\>[B]{}\hsindent{5}{}\<[5]%
\>[5]{}\Varid{to}\mathbin{∷}\Varid{a}\mathbin{→}\Varid{b}{}\<[E]%
\ColumnHook
\end{hscode}\resethooks


A fim de diminuir o número de parenteses e facilitar a legibilidade defini as funções:

bimap de tuplos (\ensuremath{(,)}):
\begin{hscode}\SaveRestoreHook
\column{B}{@{}>{\hspre}l<{\hspost}@{}}%
\column{E}{@{}>{\hspre}l<{\hspost}@{}}%
\>[B]{}\mathbf{infixr}\;\mathrm{6}\mathbin{×}{}\<[E]%
\\
\>[B]{}\mathbf{type}\;\Varid{a}\mathbin{×}\Varid{b}\mathrel{=}(\Varid{a},\Varid{b}){}\<[E]%
\\
\>[B]{}(\mathbin{×})\mathbin{∷}(\Varid{a}\mathbin{→}\Varid{b})\mathbin{→}(\Varid{c}\mathbin{→}\Varid{d})\mathbin{→}(\Varid{a},\Varid{c})\mathbin{→}(\Varid{b},\Varid{d}){}\<[E]%
\\
\>[B]{}(\mathbin{×})\mathrel{=}(\mathbin{><}){}\<[E]%
\ColumnHook
\end{hscode}\resethooks


bimap de \ensuremath{\Conid{Either}}:
\begin{hscode}\SaveRestoreHook
\column{B}{@{}>{\hspre}l<{\hspost}@{}}%
\column{E}{@{}>{\hspre}l<{\hspost}@{}}%
\>[B]{}\mathbf{infixr}\;\mathrm{4}+{}\<[E]%
\\
\>[B]{}(+)\mathbin{∷}(\Varid{a}\mathbin{→}\Varid{b})\mathbin{→}(\Varid{c}\mathbin{→}\Varid{d})\mathbin{→}\Varid{a}\mathbin{∐}\Varid{c}\mathbin{→}\Varid{b}\mathbin{∐}\Varid{d}{}\<[E]%
\\
\>[B]{}(+)\mathrel{=}(\mathbin{-|-}){}\<[E]%
\ColumnHook
\end{hscode}\resethooks


\begin{hscode}\SaveRestoreHook
\column{B}{@{}>{\hspre}l<{\hspost}@{}}%
\column{E}{@{}>{\hspre}l<{\hspost}@{}}%
\>[B]{}\mathbf{infixr}\;\mathrm{4}\mathbin{∐}{}\<[E]%
\\
\>[B]{}\mathbf{type}\;(\mathbin{∐})\mathrel{=}\Conid{Either}{}\<[E]%
\\
\>[B]{}(\mathbin{∐})\mathbin{∷}(\Varid{a}\mathbin{→}\Varid{c})\mathbin{→}(\Varid{b}\mathbin{→}\Varid{c})\mathbin{→}\Varid{a}\mathbin{∐}\Varid{b}\mathbin{→}\Varid{c}{}\<[E]%
\\
\>[B]{}(\mathbin{∐})\mathrel{=}\either{\cdot }{\cdot }{}\<[E]%
\ColumnHook
\end{hscode}\resethooks


Novamente, para simplificar a tipagem:
\begin{hscode}\SaveRestoreHook
\column{B}{@{}>{\hspre}l<{\hspost}@{}}%
\column{E}{@{}>{\hspre}l<{\hspost}@{}}%
\>[B]{}\mathbf{type}\;\Conid{BinExp}\;\Varid{d}\mathrel{=}\Conid{BinOp}\mathbin{×}\Conid{ExpAr}\;\Varid{d}\mathbin{×}\Conid{ExpAr}\;\Varid{d}{}\<[E]%
\ColumnHook
\end{hscode}\resethooks
Note que por conta de precedência \ensuremath{\Conid{BinExp}\;\Varid{d}\mathbin{≡}\Conid{BinOp}\mathbin{×}(\Conid{ExpAr}\;\Varid{d}\mathbin{×}\Conid{ExpAr}\;\Varid{d})}.

\begin{hscode}\SaveRestoreHook
\column{B}{@{}>{\hspre}l<{\hspost}@{}}%
\column{E}{@{}>{\hspre}l<{\hspost}@{}}%
\>[B]{}\mathbf{type}\;\Conid{UnExp}\;\Varid{d}\mathrel{=}\Conid{UnOp}\mathbin{×}\Conid{ExpAr}\;\Varid{d}{}\<[E]%
\ColumnHook
\end{hscode}\resethooks


Isso é uma redefinição do que o Professor definiu.
É igual excepto os símbolos mais fáceis de ler.
\begin{hscode}\SaveRestoreHook
\column{B}{@{}>{\hspre}l<{\hspost}@{}}%
\column{E}{@{}>{\hspre}l<{\hspost}@{}}%
\>[B]{}\mathbf{type}\;\Conid{OutExpAr}\;\Varid{a}\mathrel{=}()\mathbin{∐}\Varid{a}\mathbin{∐}\Conid{BinExp}\;\Varid{a}\mathbin{∐}\Conid{UnExp}\;\Varid{a}{}\<[E]%
\ColumnHook
\end{hscode}\resethooks


Vamos criar uma interpretação de \ensuremath{\Conid{ExpAr}\;\Varid{a}} como \ensuremath{\Conid{OutExpAr}\;\Varid{a}}.
Ou seja, essa interpretação é \ensuremath{\Varid{outExpAr}}.
\begin{hscode}\SaveRestoreHook
\column{B}{@{}>{\hspre}l<{\hspost}@{}}%
\column{3}{@{}>{\hspre}l<{\hspost}@{}}%
\column{17}{@{}>{\hspre}c<{\hspost}@{}}%
\column{17E}{@{}l@{}}%
\column{20}{@{}>{\hspre}l<{\hspost}@{}}%
\column{E}{@{}>{\hspre}l<{\hspost}@{}}%
\>[B]{}\mathbf{instance}\;\Conid{Interpretation}\;(\Conid{ExpAr}\;\Varid{a})\;(\Conid{OutExpAr}\;\Varid{a})\;\mathbf{where}{}\<[E]%
\\
\>[B]{}\hsindent{3}{}\<[3]%
\>[3]{}\Varid{to}\;\Conid{X}{}\<[20]%
\>[20]{}\mathrel{=}\Varid{i}_{1}\;(){}\<[E]%
\\
\>[B]{}\hsindent{3}{}\<[3]%
\>[3]{}\Varid{to}\;(\Conid{N}\;\Varid{a}{}\<[17]%
\>[17]{}){}\<[17E]%
\>[20]{}\mathrel{=}\Varid{i}_{2}\mathbin{\$}\Varid{i}_{1}\;\Varid{a}{}\<[E]%
\\
\>[B]{}\hsindent{3}{}\<[3]%
\>[3]{}\Varid{to}\;(\Conid{Bin}\;\Varid{op}\;\Varid{l}\;\Varid{r}){}\<[20]%
\>[20]{}\mathrel{=}\Varid{i}_{2}\mathbin{\$}\Varid{i}_{2}\mathbin{\$}\Varid{i}_{1}\;(\Varid{op},(\Varid{l},\Varid{r})){}\<[E]%
\\
\>[B]{}\hsindent{3}{}\<[3]%
\>[3]{}\Varid{to}\;(\Conid{Un}\;\Varid{op}\;\Varid{a}{}\<[17]%
\>[17]{}){}\<[17E]%
\>[20]{}\mathrel{=}\Varid{i}_{2}\mathbin{\$}\Varid{i}_{2}\mathbin{\$}\Varid{i}_{2}\;(\Varid{op},\Varid{a}){}\<[E]%
\ColumnHook
\end{hscode}\resethooks


Interpretamos cada símbolo \ensuremath{\Conid{BinOp}} como uma função \ensuremath{(\Varid{c},\Varid{c})\mathbin{→}\Varid{c}}.
Por exemplo, o símbolo \ensuremath{\Conid{Sum}} é interpretado como a função \ensuremath{\Varid{add}}.
\begin{hscode}\SaveRestoreHook
\column{B}{@{}>{\hspre}l<{\hspost}@{}}%
\column{3}{@{}>{\hspre}l<{\hspost}@{}}%
\column{15}{@{}>{\hspre}l<{\hspost}@{}}%
\column{E}{@{}>{\hspre}l<{\hspost}@{}}%
\>[B]{}\mathbf{instance}\;(\Conid{Num}\;\Varid{c})\mathbin{⇒}\Conid{Interpretation}\;\Conid{BinOp}\;((\Varid{c},\Varid{c})\mathbin{→}\Varid{c})\;\mathbf{where}{}\<[E]%
\\
\>[B]{}\hsindent{3}{}\<[3]%
\>[3]{}\Varid{to}\;\Conid{Sum}{}\<[15]%
\>[15]{}\mathrel{=}\Varid{add}{}\<[E]%
\\
\>[B]{}\hsindent{3}{}\<[3]%
\>[3]{}\Varid{to}\;\Conid{Product}{}\<[15]%
\>[15]{}\mathrel{=}\Varid{mul}{}\<[E]%
\ColumnHook
\end{hscode}\resethooks


Na nossa linguagem mais usual:
\begin{hscode}\SaveRestoreHook
\column{B}{@{}>{\hspre}l<{\hspost}@{}}%
\column{E}{@{}>{\hspre}l<{\hspost}@{}}%
\>[B]{}out_{\Conid{BinOp}}\mathbin{∷}\Conid{Num}\;\Varid{c}\mathbin{⇒}\Conid{BinOp}\mathbin{→}(\Varid{c}\mathbin{×}\Varid{c})\mathbin{→}\Varid{c}{}\<[E]%
\\
\>[B]{}out_{\Conid{BinOp}}\mathrel{=}\Varid{to}\mathbin{∷}(\Conid{Num}\;\Varid{c})\mathbin{⇒}\Conid{BinOp}\mathbin{→}(\Varid{c}\mathbin{×}\Varid{c}\mathbin{→}\Varid{c}){}\<[E]%
\ColumnHook
\end{hscode}\resethooks


Interpretamos cada símbolo \ensuremath{\Conid{UnOp}} como uma função \ensuremath{\Varid{c}\mathbin{→}\Varid{c}} onde \ensuremath{\Varid{c}}
é da classe \ensuremath{\Conid{Floating}}.
\begin{hscode}\SaveRestoreHook
\column{B}{@{}>{\hspre}l<{\hspost}@{}}%
\column{3}{@{}>{\hspre}l<{\hspost}@{}}%
\column{14}{@{}>{\hspre}l<{\hspost}@{}}%
\column{E}{@{}>{\hspre}l<{\hspost}@{}}%
\>[B]{}\mathbf{instance}\;(\Conid{Floating}\;\Varid{c})\mathbin{⇒}\Conid{Interpretation}\;\Conid{UnOp}\;(\Varid{c}\mathbin{→}\Varid{c})\;\mathbf{where}{}\<[E]%
\\
\>[B]{}\hsindent{3}{}\<[3]%
\>[3]{}\Varid{to}\;\Conid{Negate}{}\<[14]%
\>[14]{}\mathrel{=}\Varid{negate}{}\<[E]%
\\
\>[B]{}\hsindent{3}{}\<[3]%
\>[3]{}\Varid{to}\;\Conid{E}{}\<[14]%
\>[14]{}\mathrel{=}\Varid{\Conid{Prelude}.exp}{}\<[E]%
\ColumnHook
\end{hscode}\resethooks


Na nossa linguagem mais usual
\begin{hscode}\SaveRestoreHook
\column{B}{@{}>{\hspre}l<{\hspost}@{}}%
\column{E}{@{}>{\hspre}l<{\hspost}@{}}%
\>[B]{}out_{\Conid{UnOp}}\mathbin{∷}\Conid{Floating}\;\Varid{c}\mathbin{⇒}\Conid{UnOp}\mathbin{→}(\Varid{c}\mathbin{→}\Varid{c}){}\<[E]%
\\
\>[B]{}out_{\Conid{UnOp}}\mathrel{=}\Varid{to}\mathbin{∷}(\Conid{Floating}\;\Varid{c})\mathbin{⇒}\Conid{UnOp}\mathbin{→}(\Varid{c}\mathbin{→}\Varid{c}){}\<[E]%
\ColumnHook
\end{hscode}\resethooks


\begin{hscode}\SaveRestoreHook
\column{B}{@{}>{\hspre}l<{\hspost}@{}}%
\column{E}{@{}>{\hspre}l<{\hspost}@{}}%
\>[B]{}\Varid{outExpAr}\mathbin{∷}\Conid{ExpAr}\;\Varid{a}\mathbin{→}\Conid{OutExpAr}\;\Varid{a}{}\<[E]%
\\
\>[B]{}\Varid{outExpAr}\mathrel{=}\Varid{to}\mathbin{∷}\Conid{ExpAr}\;\Varid{a}\mathbin{→}\Conid{OutExpAr}\;\Varid{a}{}\<[E]%
\\[\blanklineskip]%
\>[B]{}\Varid{recExpAr}\mathbin{∷}(\Varid{a}\mathbin{→}\Varid{e})\mathbin{→}\Varid{b}\mathbin{∐}\Varid{c}\mathbin{∐}\Varid{d}\mathbin{×}\Varid{a}\mathbin{×}\Varid{a}\mathbin{∐}\Varid{g}\mathbin{×}\Varid{a}\mathbin{→}\Varid{b}\mathbin{∐}\Varid{c}\mathbin{∐}\Varid{d}\mathbin{×}\Varid{e}\mathbin{×}\Varid{e}\mathbin{∐}\Varid{g}\mathbin{×}\Varid{e}{}\<[E]%
\\
\>[B]{}\Varid{recExpAr}\;\Varid{f}\mathrel{=}\Varid{baseExpAr}\;\Varid{id}\;\Varid{id}\;\Varid{id}\;\Varid{f}\;\Varid{f}\;\Varid{id}\;\Varid{f}{}\<[E]%
\\[\blanklineskip]%
\>[B]{}\Varid{g\char95 eval\char95 exp}\mathbin{∷}\Conid{Floating}\;\Varid{c}\mathbin{⇒}\Varid{c}\mathbin{→}\Varid{b}\mathbin{∐}\Varid{c}\mathbin{∐}\Conid{BinOp}\mathbin{×}\Varid{c}\mathbin{×}\Varid{c}\mathbin{∐}\Conid{UnOp}\mathbin{×}\Varid{c}\mathbin{→}\Varid{c}{}\<[E]%
\\
\>[B]{}\Varid{g\char95 eval\char95 exp}\;\Varid{a}\mathrel{=}\const{\Varid{a}}\mathbin{∐}\Varid{id}\mathbin{∐}\Varid{ap}\mathbin{\circ}(out_{\Conid{BinOp}}\mathbin{×}\Varid{id})\mathbin{∐}\Varid{ap}\mathbin{\circ}(out_{\Conid{UnOp}}\mathbin{×}\Varid{id}){}\<[E]%
\ColumnHook
\end{hscode}\resethooks


Nós optimizamos 4 casos e para os outros usamos \ensuremath{\Varid{outExpAr}}:
\begin{hscode}\SaveRestoreHook
\column{B}{@{}>{\hspre}l<{\hspost}@{}}%
\column{3}{@{}>{\hspre}l<{\hspost}@{}}%
\column{9}{@{}>{\hspre}l<{\hspost}@{}}%
\column{18}{@{}>{\hspre}l<{\hspost}@{}}%
\column{25}{@{}>{\hspre}l<{\hspost}@{}}%
\column{30}{@{}>{\hspre}c<{\hspost}@{}}%
\column{30E}{@{}l@{}}%
\column{33}{@{}>{\hspre}l<{\hspost}@{}}%
\column{E}{@{}>{\hspre}l<{\hspost}@{}}%
\>[B]{}\Varid{clean}\mathbin{∷}(\Conid{Eq}\;\Varid{a},\Conid{Num}\;\Varid{a})\mathbin{⇒}\Conid{ExpAr}\;\Varid{a}\mathbin{→}\Conid{OutExpAr}\;\Varid{a}{}\<[E]%
\\
\>[B]{}\Varid{clean}\mathrel{=}\lambda \mathbf{case}{}\<[E]%
\\
\>[B]{}\hsindent{3}{}\<[3]%
\>[3]{}(\Conid{Un}\;{}\<[9]%
\>[9]{}\Conid{E}\;{}\<[18]%
\>[18]{}(\Conid{N}\;\mathrm{0}){}\<[30]%
\>[30]{}){}\<[30E]%
\>[33]{}\mathbin{→}\Varid{tag}\;\mathrm{1}{}\<[E]%
\\
\>[B]{}\hsindent{3}{}\<[3]%
\>[3]{}(\Conid{Un}\;{}\<[9]%
\>[9]{}\Conid{Negate}\;{}\<[18]%
\>[18]{}(\Conid{N}\;\mathrm{0}){}\<[30]%
\>[30]{}){}\<[30E]%
\>[33]{}\mathbin{→}\Varid{tag}\;\mathrm{0}{}\<[E]%
\\
\>[B]{}\hsindent{3}{}\<[3]%
\>[3]{}(\Conid{Bin}\;{}\<[9]%
\>[9]{}\Conid{Product}\;{}\<[18]%
\>[18]{}(\Conid{N}\;\mathrm{0})\;{}\<[25]%
\>[25]{}\anonymous {}\<[30]%
\>[30]{}){}\<[30E]%
\>[33]{}\mathbin{→}\Varid{tag}\;\mathrm{0}{}\<[E]%
\\
\>[B]{}\hsindent{3}{}\<[3]%
\>[3]{}(\Conid{Bin}\;{}\<[9]%
\>[9]{}\Conid{Product}\;{}\<[18]%
\>[18]{}\anonymous \;{}\<[25]%
\>[25]{}(\Conid{N}\;\mathrm{0})){}\<[33]%
\>[33]{}\mathbin{→}\Varid{tag}\;\mathrm{0}{}\<[E]%
\\
\>[B]{}\hsindent{3}{}\<[3]%
\>[3]{}\Varid{a}{}\<[33]%
\>[33]{}\mathbin{→}\Varid{outExpAr}\;\Varid{a}{}\<[E]%
\\
\>[B]{}\hsindent{3}{}\<[3]%
\>[3]{}\mathbf{where}\;\Varid{tag}\mathrel{=}\Varid{i}_{2}\mathbin{\circ}\Varid{i}_{1}{}\<[E]%
\ColumnHook
\end{hscode}\resethooks


\begin{hscode}\SaveRestoreHook
\column{B}{@{}>{\hspre}l<{\hspost}@{}}%
\column{E}{@{}>{\hspre}l<{\hspost}@{}}%
\>[B]{}\Varid{gopt}\mathbin{∷}\Conid{Floating}\;\Varid{a}\mathbin{⇒}\Varid{a}\mathbin{→}()\mathbin{∐}\Varid{a}\mathbin{∐}\Conid{BinOp}\mathbin{×}\Varid{a}\mathbin{×}\Varid{a}\mathbin{∐}\Conid{UnOp}\mathbin{×}\Varid{a}\mathbin{→}\Varid{a}{}\<[E]%
\\
\>[B]{}\Varid{gopt}\mathrel{=}\Varid{g\char95 eval\char95 exp}{}\<[E]%
\ColumnHook
\end{hscode}\resethooks


Baseado na função \ensuremath{\Varid{dup}} definida em \texttt{Cp.hs}:
\begin{hscode}\SaveRestoreHook
\column{B}{@{}>{\hspre}l<{\hspost}@{}}%
\column{E}{@{}>{\hspre}l<{\hspost}@{}}%
\>[B]{}\mathbf{type}\;\Conid{Dup}\;\Varid{d}\mathrel{=}\Varid{d}\mathbin{×}\Varid{d}{}\<[E]%
\ColumnHook
\end{hscode}\resethooks

\begin{hscode}\SaveRestoreHook
\column{B}{@{}>{\hspre}l<{\hspost}@{}}%
\column{13}{@{}>{\hspre}l<{\hspost}@{}}%
\column{E}{@{}>{\hspre}l<{\hspost}@{}}%
\>[B]{}\mathbf{type}\;\Conid{Bin}\;\Varid{d}{}\<[13]%
\>[13]{}\mathrel{=}\Conid{BinOp}\mathbin{×}\Conid{Dup}\;\Varid{d}{}\<[E]%
\\
\>[B]{}\mathbf{type}\;\Conid{Un}\;\Varid{d}{}\<[13]%
\>[13]{}\mathrel{=}\Conid{UnOp}\mathbin{×}\Varid{d}{}\<[E]%
\ColumnHook
\end{hscode}\resethooks


A fim de criar código mais elegante extrai repetições de
extrai o que se repetia nas funções \ensuremath{\Varid{sd\char95 gen}} e \ensuremath{\Varid{ad\char95 gen}} do Tiago.
\begin{hscode}\SaveRestoreHook
\column{B}{@{}>{\hspre}l<{\hspost}@{}}%
\column{3}{@{}>{\hspre}l<{\hspost}@{}}%
\column{11}{@{}>{\hspre}l<{\hspost}@{}}%
\column{12}{@{}>{\hspre}l<{\hspost}@{}}%
\column{E}{@{}>{\hspre}l<{\hspost}@{}}%
\>[B]{}{bin}_{aux}\;{}\;{}\;{}\mathbin{∷}(\Varid{t}\mathbin{→}\Varid{t}\mathbin{→}\Varid{t})\mathbin{→}(\Varid{t}\mathbin{→}\Varid{t}\mathbin{→}\Varid{t})\mathbin{→}(\Conid{BinOp},\Conid{Dup}\;(\Conid{Dup}\;\Varid{t}))\mathbin{→}\Conid{Dup}\;\Varid{t}{}\<[E]%
\\
\>[B]{}{bin}_{aux}\;\mathbin{♢}\;\mathbin{□}\;(\Varid{op},((\Varid{e}_{1},\Varid{d}_{1}),(\Varid{e}_{2},\Varid{d}_{2})))\mathrel{=}\mathbf{case}\;\Varid{op}\;\mathbf{of}{}\<[E]%
\\
\>[B]{}\hsindent{3}{}\<[3]%
\>[3]{}\Conid{Sum}{}\<[12]%
\>[12]{}\mathbin{→}(\Varid{e}_{1}\mathbin{♢}\Varid{e}_{2},\Varid{d}_{1}\mathbin{♢}\Varid{d}_{2}){}\<[E]%
\\
\>[B]{}\hsindent{3}{}\<[3]%
\>[3]{}\Conid{Product}{}\<[12]%
\>[12]{}\mathbin{→}(\Varid{e}_{1}\mathbin{□}\Varid{e}_{2},(\Varid{e}_{1}\mathbin{□}\Varid{d}_{2})\mathbin{♢}(\Varid{d}_{1}\mathbin{□}\Varid{e}_{2})){}\<[E]%
\\[\blanklineskip]%
\>[B]{}{un}_{aux}\;{}\;{}\;{}\;{}\mathbin{∷}(\Varid{a}\mathbin{→}\Varid{b})\mathbin{→}(\Varid{b}\mathbin{→}\Varid{a}\mathbin{→}\Varid{b})\mathbin{→}(\Varid{a}\mathbin{→}\Varid{b})\mathbin{→}(\Conid{UnOp},\Conid{Dup}\;\Varid{a})\mathbin{→}\Conid{Dup}\;\Varid{b}{}\<[E]%
\\
\>[B]{}{un}_{aux}\;\circleddash\;\mathbin{⊛}\;\boxminus\;(\Varid{op},(\Varid{e},\Varid{d}))\mathrel{=}\mathbf{case}\;\Varid{op}\;\mathbf{of}{}\<[E]%
\\
\>[B]{}\hsindent{3}{}\<[3]%
\>[3]{}\Conid{Negate}{}\<[11]%
\>[11]{}\mathbin{→}(\circleddash\;\Varid{e},\circleddash\;\Varid{d}){}\<[E]%
\\
\>[B]{}\hsindent{3}{}\<[3]%
\>[3]{}\Conid{E}{}\<[11]%
\>[11]{}\mathbin{→}(\boxminus\;\Varid{e},(\boxminus\;\Varid{e})\mathbin{⊛}\Varid{d}){}\<[E]%
\ColumnHook
\end{hscode}\resethooks


\begin{hscode}\SaveRestoreHook
\column{B}{@{}>{\hspre}l<{\hspost}@{}}%
\column{3}{@{}>{\hspre}l<{\hspost}@{}}%
\column{8}{@{}>{\hspre}l<{\hspost}@{}}%
\column{E}{@{}>{\hspre}l<{\hspost}@{}}%
\>[B]{}\Varid{sd\char95 gen}\mathbin{∷}\Conid{Floating}\;\Varid{a}\mathbin{⇒}()\mathbin{∐}\Varid{a}\mathbin{∐}\Conid{Bin}\;(\Conid{Dup}\;(\Conid{ExpAr}\;\Varid{a}))\mathbin{∐}\Conid{Un}\;(\Conid{Dup}\;(\Conid{ExpAr}\;\Varid{a}))\mathbin{→}\Conid{Dup}\;(\Conid{ExpAr}\;\Varid{a}){}\<[E]%
\\
\>[B]{}\Varid{sd\char95 gen}\mathrel{=}\Varid{f}\mathbin{∐}\Varid{g}\mathbin{∐}\Varid{h}\mathbin{∐}\Varid{k}\;\mathbf{where}{}\<[E]%
\\
\>[B]{}\hsindent{3}{}\<[3]%
\>[3]{}\Varid{f}{}\<[8]%
\>[8]{}\mathrel{=}\const{(\Conid{X},\Conid{N}\;\mathrm{1})}{}\<[E]%
\\
\>[B]{}\hsindent{3}{}\<[3]%
\>[3]{}\Varid{g}\;\Varid{a}{}\<[8]%
\>[8]{}\mathrel{=}(\Conid{N}\;\Varid{a},\Conid{N}\;\mathrm{0}){}\<[E]%
\\
\>[B]{}\hsindent{3}{}\<[3]%
\>[3]{}\Varid{h}{}\<[8]%
\>[8]{}\mathrel{=}{bin}_{aux}\;(\Conid{Bin}\;\Conid{Sum})\;(\Conid{Bin}\;\Conid{Product}){}\<[E]%
\\
\>[B]{}\hsindent{3}{}\<[3]%
\>[3]{}\Varid{k}{}\<[8]%
\>[8]{}\mathrel{=}{un}_{aux}\;(\Conid{Un}\;\Conid{Negate})\;(\Conid{Bin}\;\Conid{Product})\;(\Conid{Un}\;\Conid{E}){}\<[E]%
\\[\blanklineskip]%
\>[B]{}\Varid{ad\char95 gen}\mathbin{∷}\Conid{Floating}\;\Varid{a}\mathbin{⇒}\Varid{a}\mathbin{→}()\mathbin{∐}\Varid{a}\mathbin{∐}(\Conid{BinOp},\Conid{Dup}\;(\Conid{Dup}\;\Varid{a}))\mathbin{∐}(\Conid{UnOp},\Conid{Dup}\;\Varid{a})\mathbin{→}\Conid{Dup}\;\Varid{a}{}\<[E]%
\\
\>[B]{}\Varid{ad\char95 gen}\;\Varid{x}\mathrel{=}\Varid{f}\mathbin{∐}\Varid{g}\mathbin{∐}\Varid{h}\mathbin{∐}\Varid{k}\;\mathbf{where}{}\<[E]%
\\
\>[B]{}\hsindent{3}{}\<[3]%
\>[3]{}\Varid{f}{}\<[8]%
\>[8]{}\mathrel{=}\const{(\Varid{x},\mathrm{1})}{}\<[E]%
\\
\>[B]{}\hsindent{3}{}\<[3]%
\>[3]{}\Varid{g}\;\Varid{a}{}\<[8]%
\>[8]{}\mathrel{=}(\Varid{a},\mathrm{0}){}\<[E]%
\\
\>[B]{}\hsindent{3}{}\<[3]%
\>[3]{}\Varid{h}{}\<[8]%
\>[8]{}\mathrel{=}{bin}_{aux}\;(\mathbin{+})\;(\mathbin{*}){}\<[E]%
\\
\>[B]{}\hsindent{3}{}\<[3]%
\>[3]{}\Varid{k}{}\<[8]%
\>[8]{}\mathrel{=}{un}_{aux}\;\Varid{negate}\;(\mathbin{*})\;\Varid{expd}{}\<[E]%
\ColumnHook
\end{hscode}\resethooks

\subsection{Problema 2}
\begin{hscode}\SaveRestoreHook
\column{B}{@{}>{\hspre}l<{\hspost}@{}}%
\column{E}{@{}>{\hspre}l<{\hspost}@{}}%
\>[B]{}\Varid{loop}\mathbin{∷}\Conid{Integral}\;\Varid{c}\mathbin{⇒}(\Varid{c},\Varid{c},\Varid{c})\mathbin{→}(\Varid{c},\Varid{c},\Varid{c}){}\<[E]%
\\
\>[B]{}\Varid{loop}\mathrel{=}\Varid{g}\;\mathbf{where}\;\Varid{g}\;(\Varid{a},\Varid{b},\Varid{c})\mathrel{=}(\Varid{div}\;(\Varid{a}\mathbin{*}\Varid{b})\;\Varid{c},\Varid{b}\mathbin{+}\mathrm{4},\Varid{c}\mathbin{+}\mathrm{1}){}\<[E]%
\\[\blanklineskip]%
\>[B]{}\Varid{inic}\mathbin{∷}(\Conid{Num}\;\Varid{a},\Conid{Num}\;\Varid{b},\Conid{Num}\;\Varid{c})\mathbin{⇒}(\Varid{a},\Varid{b},\Varid{c}){}\<[E]%
\\
\>[B]{}\Varid{inic}\mathrel{=}(\mathrm{1},\mathrm{2},\mathrm{2}){}\<[E]%
\\[\blanklineskip]%
\>[B]{}\Varid{prj}\mathbin{∷}(\Varid{a},\Varid{b},\Varid{c})\mathbin{→}\Varid{a}{}\<[E]%
\\
\>[B]{}\Varid{prj}\mathrel{=}\Varid{p}\;\mathbf{where}\;\Varid{p}\;(\Varid{a},\anonymous ,\anonymous )\mathrel{=}\Varid{a}{}\<[E]%
\\[\blanklineskip]%
\>[B]{}\Varid{cat}\mathbin{∷}(\Conid{Integral}\;\Varid{c1},\Conid{Integral}\;\Varid{c2})\mathbin{⇒}\Varid{c1}\mathbin{→}\Varid{c2}{}\<[E]%
\\
\>[B]{}\Varid{cat}\mathrel{=}\Varid{prj}\mathbin{\circ}\Varid{for}\;\Varid{loop}\;\Varid{inic}{}\<[E]%
\ColumnHook
\end{hscode}\resethooks

\subsection{Problema 3}
É interessante ver que podemos ver \ensuremath{\Varid{calcLine}} como um hilomorfismo.

A ideia que levou a isso parte da definição alternativa
\ensuremath{\Varid{calcLine}\mathrel{=}\Varid{zipWithM}\;\Varid{linear1d}}.


Sabemos que:
\begin{hscode}\SaveRestoreHook
\column{B}{@{}>{\hspre}l<{\hspost}@{}}%
\column{E}{@{}>{\hspre}l<{\hspost}@{}}%
\>[B]{}\Varid{zipWithM}\mathbin{∷}(\Conid{Applicative}\;\Varid{m})\mathbin{⇒}(\Varid{a}\mathbin{→}\Varid{b}\mathbin{→}\Varid{m}\;\Varid{c})\mathbin{→}[\mskip1.5mu \Varid{a}\mskip1.5mu]\mathbin{→}[\mskip1.5mu \Varid{b}\mskip1.5mu]\mathbin{→}\Varid{m}\;[\mskip1.5mu \Varid{c}\mskip1.5mu]{}\<[E]%
\\
\>[B]{}\Varid{zipWithM}\;\Varid{f}\;\Varid{xs}\;\Varid{ys}\mathrel{=}\Varid{sequenceA}\;(\Varid{zipWith}\;\Varid{f}\;\Varid{xs}\;\Varid{ys}){}\<[E]%
\ColumnHook
\end{hscode}\resethooks


Percebi que podia escrever uma função (\ensuremath{\curry{\Varid{zip}}}):
\begin{hscode}\SaveRestoreHook
\column{B}{@{}>{\hspre}l<{\hspost}@{}}%
\column{E}{@{}>{\hspre}l<{\hspost}@{}}%
\>[B]{}\Varid{zip'}\mathbin{∷}[\mskip1.5mu \Varid{a}\mskip1.5mu]\mathbin{×}[\mskip1.5mu \Varid{b}\mskip1.5mu]\mathbin{→}[\mskip1.5mu \Varid{a}\mathbin{×}\Varid{b}\mskip1.5mu]{}\<[E]%
\\
\>[B]{}\Varid{zip'}\mathrel{=}\ana{out_{A^{*}\times B^{*}}}_{[\mskip1.5mu \mskip1.5mu]}{}\<[E]%
\ColumnHook
\end{hscode}\resethooks


Desde que transforme os pares de lista de uma forma que respeite
o funcionamento de \ensuremath{\Varid{zipWith}} que será descrito em seguida dessa definição:
\begin{hscode}\SaveRestoreHook
\column{B}{@{}>{\hspre}l<{\hspost}@{}}%
\column{3}{@{}>{\hspre}l<{\hspost}@{}}%
\column{21}{@{}>{\hspre}l<{\hspost}@{}}%
\column{E}{@{}>{\hspre}l<{\hspost}@{}}%
\>[B]{}out_{A^{*}\times B^{*}}\mathbin{∷}[\mskip1.5mu \Varid{a}\mskip1.5mu]\mathbin{×}[\mskip1.5mu \Varid{b}\mskip1.5mu]\mathbin{→}\Conid{Either}\;()\;((\Varid{a}\mathbin{×}\Varid{b})\mathbin{×}([\mskip1.5mu \Varid{a}\mskip1.5mu]\mathbin{×}[\mskip1.5mu \Varid{b}\mskip1.5mu])){}\<[E]%
\\
\>[B]{}out_{A^{*}\times B^{*}}\mathrel{=}\lambda \mathbf{case}{}\<[E]%
\\
\>[B]{}\hsindent{3}{}\<[3]%
\>[3]{}([\mskip1.5mu \mskip1.5mu],\anonymous ){}\<[21]%
\>[21]{}\mathbin{→}\Varid{i}_{1}\;(){}\<[E]%
\\
\>[B]{}\hsindent{3}{}\<[3]%
\>[3]{}(\anonymous ,[\mskip1.5mu \mskip1.5mu]){}\<[21]%
\>[21]{}\mathbin{→}\Varid{i}_{1}\;(){}\<[E]%
\\
\>[B]{}\hsindent{3}{}\<[3]%
\>[3]{}(\Varid{a}\mathbin{:}\Varid{as},\Varid{b}\mathbin{:}\Varid{bs}){}\<[21]%
\>[21]{}\mathbin{→}\Varid{i}_{2}\;((\Varid{a},\Varid{b}),(\Varid{as},\Varid{bs})){}\<[E]%
\ColumnHook
\end{hscode}\resethooks


\ensuremath{\Varid{zipWith}} pega uma função (de aridade 2), por exemplo, $f$, e
duas listas (digamos $a$ e $b$) e devolve uma lista (digamos $c$)
onde $c[i] = f(a[i],b[i])$ para todo $0≤i≤\ensuremath{\Varid{min}\;(\Varid{length}\;\Varid{a},\Varid{length}\;\Varid{b})}$.


Ora, então posso pegar uma função curried e pegar uma par de listas.
Transformo o par de listas numa lista de pares com \ensuremath{out_{A^{*}\times B^{*}}} e aplico a
função argumento em cada um dos pares. Logo tenho a seguinte definição:
\begin{hscode}\SaveRestoreHook
\column{B}{@{}>{\hspre}l<{\hspost}@{}}%
\column{E}{@{}>{\hspre}l<{\hspost}@{}}%
\>[B]{}\Varid{zipWith'}\mathbin{∷}((\Varid{a}\mathbin{×}\Varid{b})\mathbin{→}\Varid{c})\mathbin{→}([\mskip1.5mu \Varid{a}\mskip1.5mu]\mathbin{×}[\mskip1.5mu \Varid{b}\mskip1.5mu])\mathbin{→}[\mskip1.5mu \Varid{c}\mskip1.5mu]{}\<[E]%
\\
\>[B]{}\Varid{zipWith'}\;\Varid{f}\mathrel{=}T_{[]}\Varid{f}\mathbin{\circ}\Varid{zip'}{}\<[E]%
\ColumnHook
\end{hscode}\resethooks


A próxima etapa é baseada nas seguintes definições
\begin{hscode}\SaveRestoreHook
\column{B}{@{}>{\hspre}l<{\hspost}@{}}%
\column{E}{@{}>{\hspre}l<{\hspost}@{}}%
\>[B]{}\Varid{sequenceA}\mathbin{∷}\Conid{Applicative}\;\Varid{f}\mathbin{⇒}\Varid{t}\;(\Varid{f}\;\Varid{a})\mathbin{→}\Varid{f}\;(\Varid{t}\;\Varid{a}){}\<[E]%
\\
\>[B]{}\Varid{sequenceA}\mathrel{=}\Varid{traverse}\;\Varid{id}{}\<[E]%
\\[\blanklineskip]%
\>[B]{}\Varid{traverse}\mathbin{∷}\Conid{Applicative}\;\Varid{f}\mathbin{⇒}(\Varid{a}\mathbin{→}\Varid{f}\;\Varid{b})\mathbin{→}\Varid{t}\;\Varid{a}\mathbin{→}\Varid{f}\;(\Varid{t}\;\Varid{b}){}\<[E]%
\\
\>[B]{}\Varid{traverse}\;\Varid{f}\mathrel{=}\Varid{sequenceA}\mathbin{\circ}T_{[]}\Varid{f}{}\<[E]%
\ColumnHook
\end{hscode}\resethooks


Ora, vamos ver como \ensuremath{\Varid{traverse}} é definido para listas
\begin{hscode}\SaveRestoreHook
\column{B}{@{}>{\hspre}l<{\hspost}@{}}%
\column{5}{@{}>{\hspre}l<{\hspost}@{}}%
\column{7}{@{}>{\hspre}l<{\hspost}@{}}%
\column{E}{@{}>{\hspre}l<{\hspost}@{}}%
\>[B]{}\mathbf{instance}\;\Conid{Traversable}\;[\mskip1.5mu \mskip1.5mu]\;\mathbf{where}{}\<[E]%
\\
\>[B]{}\hsindent{5}{}\<[5]%
\>[5]{}\Varid{traverse}\;\Varid{f}\mathrel{=}\Varid{foldr}\;\Varid{cons\char95 f}\;(\Varid{pure}\;[\mskip1.5mu \mskip1.5mu]){}\<[E]%
\\
\>[5]{}\hsindent{2}{}\<[7]%
\>[7]{}\mathbf{where}\;\Varid{cons\char95 f}\;\Varid{x}\;\Varid{ys}\mathrel{=}\Varid{liftA2}\;(\mathbin{:})\;(\Varid{f}\;\Varid{x})\;\Varid{ys}{}\<[E]%
\ColumnHook
\end{hscode}\resethooks


Vou criar um \ensuremath{\Varid{sequenceA'}} (será uma versão menos genérica de \ensuremath{\Varid{sequenceA}} uma vez que estamos
sendo específicos no trabalho com listas).
\begin{hscode}\SaveRestoreHook
\column{B}{@{}>{\hspre}l<{\hspost}@{}}%
\column{3}{@{}>{\hspre}l<{\hspost}@{}}%
\column{E}{@{}>{\hspre}l<{\hspost}@{}}%
\>[B]{}\Varid{sequenceA}\mathrel{=}\Varid{traverse}\;\Varid{id}\mathrel{=}\Varid{foldr}\;\Varid{cons\char95 f}\;(\Varid{pure}\;[\mskip1.5mu \mskip1.5mu])\;\mathbf{where}{}\<[E]%
\\
\>[B]{}\hsindent{3}{}\<[3]%
\>[3]{}\Varid{cons\char95 f}\;\Varid{x}\;\Varid{ys}\mathrel{=}\Varid{liftA2}\;(\mathbin{:})\;\Varid{x}\;\Varid{ys}{}\<[E]%
\ColumnHook
\end{hscode}\resethooks


Já fizemos o catamorfismo para \ensuremath{\Varid{foldr}} nas aulas:
\begin{hscode}\SaveRestoreHook
\column{B}{@{}>{\hspre}l<{\hspost}@{}}%
\column{E}{@{}>{\hspre}l<{\hspost}@{}}%
\>[B]{}\Varid{foldrC}\mathbin{::}(\Varid{a}\to \Varid{b}\to \Varid{b})\to \Varid{b}\to [\mskip1.5mu \Varid{a}\mskip1.5mu]\to \Varid{b}{}\<[E]%
\\
\>[B]{}\Varid{foldrC}\;\Varid{f}\;\Varid{u}\mathrel{=}\cata{\either{\const{\Varid{u}}}{{\uncurry{\Varid{f}}}}}_{[\mskip1.5mu \mskip1.5mu]}{}\<[E]%
\ColumnHook
\end{hscode}\resethooks


Então temos:
\begin{hscode}\SaveRestoreHook
\column{B}{@{}>{\hspre}l<{\hspost}@{}}%
\column{3}{@{}>{\hspre}l<{\hspost}@{}}%
\column{E}{@{}>{\hspre}l<{\hspost}@{}}%
\>[B]{}\Varid{sequenceA'}\mathbin{∷}\Conid{Applicative}\;\Varid{f}\mathbin{⇒}[\mskip1.5mu \Varid{f}\;\Varid{a}\mskip1.5mu]\mathbin{→}\Varid{f}\;[\mskip1.5mu \Varid{a}\mskip1.5mu]{}\<[E]%
\\
\>[B]{}\Varid{sequenceA'}\mathrel{=}\cata{\either{\Varid{b}}{{\uncurry{\Varid{g}}}}}_{[\mskip1.5mu \mskip1.5mu]}\;\mathbf{where}{}\<[E]%
\\
\>[B]{}\hsindent{3}{}\<[3]%
\>[3]{}\Varid{b}\mathrel{=}\const{\Varid{pure}\;[\mskip1.5mu \mskip1.5mu]}{}\<[E]%
\\
\>[B]{}\hsindent{3}{}\<[3]%
\>[3]{}\Varid{g}\;\Varid{x}\;\Varid{ys}\mathrel{=}\Varid{liftA2}\;(\mathbin{:})\;\Varid{x}\;\Varid{ys}{}\<[E]%
\ColumnHook
\end{hscode}\resethooks


Sabemos que, em \ensuremath{\Conid{Applicative}\;((\mathbin{→})\;\Varid{r})}, \ensuremath{\Varid{pure}\mathrel{=}\text{\ttfamily 'const'}} e
\ensuremath{\Varid{liftA2}\;\Varid{q}\;\Varid{f}\;\Varid{g}\;\Varid{x}\mathrel{=}\Varid{q}\;(\Varid{f}\;\Varid{x})\;(\Varid{g}\;\Varid{x})}. Logo:
\begin{hscode}\SaveRestoreHook
\column{B}{@{}>{\hspre}l<{\hspost}@{}}%
\column{3}{@{}>{\hspre}l<{\hspost}@{}}%
\column{E}{@{}>{\hspre}l<{\hspost}@{}}%
\>[B]{}\Varid{sequenceA'}\mathbin{∷}[\mskip1.5mu \Varid{a}\mathbin{→}\Varid{b}\mskip1.5mu]\mathbin{→}\Varid{a}\mathbin{→}[\mskip1.5mu \Varid{b}\mskip1.5mu]{}\<[E]%
\\
\>[B]{}\Varid{sequenceA'}\mathrel{=}\cata{\either{\Varid{b}}{{\uncurry{\Varid{g}}}}}_{[\mskip1.5mu \mskip1.5mu]}\;\mathbf{where}{}\<[E]%
\\
\>[B]{}\hsindent{3}{}\<[3]%
\>[3]{}\Varid{b}\mathrel{=}\const{\const{[\mskip1.5mu \mskip1.5mu]}}{}\<[E]%
\\
\>[B]{}\hsindent{3}{}\<[3]%
\>[3]{}\Varid{g}\;\Varid{x}\;\Varid{ys}\mathrel{=}(\lambda \Varid{z}\mathbin{→}\Varid{x}\;\Varid{z}\mathbin{:}\Varid{ys}\;\Varid{z}){}\<[E]%
\ColumnHook
\end{hscode}\resethooks




Lembre que \ensuremath{\Varid{zipWithM'}} como vimos recebia duas listas. No nosso caso
essas duas listas (digamos \ensuremath{\Varid{xs}} e \ensuremath{\Varid{ys}}) estão em um só argumento \ensuremath{\Varid{t}\mathrel{=}(\Varid{xs},\Varid{ys})}
Agora, estamos em condições de escrever:
\def\commentbegin{\quad\{\ }
\def\commentend{\}}
\begin{hscode}\SaveRestoreHook
\column{B}{@{}>{\hspre}l<{\hspost}@{}}%
\column{17}{@{}>{\hspre}l<{\hspost}@{}}%
\column{100}{@{}>{\hspre}l<{\hspost}@{}}%
\column{E}{@{}>{\hspre}l<{\hspost}@{}}%
\>[B]{}\Varid{zipWithM'}\;\Varid{f}\;\Varid{t}{}\<[17]%
\>[17]{}\mathrel{=}\Varid{sequenceA'}\;(\Varid{zipWith'}\;\Varid{f}\;\Varid{t}){}\<[E]%
\\[\blanklineskip]%
\>[B]{}\mbox{\commentbegin  \ensuremath{(\mathbin{\circ})\;\Varid{f}\;\Varid{g}\mathrel{=}\lambda \Varid{x}\to \Varid{f}\;(\Varid{g}\;\Varid{x})}  \commentend}{}\<[E]%
\\[\blanklineskip]%
\>[B]{}\Varid{zipWithM'}\;\Varid{f}{}\<[17]%
\>[17]{}\mathrel{=}\Varid{sequenceA'}\mathbin{\circ}\Varid{zipWith'}\;\Varid{f}{}\<[E]%
\\[\blanklineskip]%
\>[B]{}\mbox{\commentbegin  Def-\ensuremath{\Varid{zipWith'}}  \commentend}{}\<[E]%
\\[\blanklineskip]%
\>[B]{}\Varid{zipWithM'}\;\Varid{f}{}\<[17]%
\>[17]{}\mathrel{=}\Varid{sequenceA'}\mathbin{\circ}(T_{[]}\Varid{f}\mathbin{\circ}\Varid{zip'}){}\<[E]%
\\[\blanklineskip]%
\>[B]{}\mbox{\commentbegin  Assoc-comp  \commentend}{}\<[E]%
\\[\blanklineskip]%
\>[B]{}\Varid{zipWithM'}\;\Varid{f}{}\<[17]%
\>[17]{}\mathrel{=}(\Varid{sequenceA'}\mathbin{\circ}T_{[]}\Varid{f})\mathbin{\circ}\Varid{zip'}{}\<[E]%
\\[\blanklineskip]%
\>[B]{}\mbox{\commentbegin  Def-\ensuremath{\Varid{sequenceA'}}  \commentend}{}\<[E]%
\\[\blanklineskip]%
\>[B]{}\Varid{zipWithM'}\;\Varid{f}{}\<[17]%
\>[17]{}\mathrel{=}(\cata{\either{\const{\const{[\mskip1.5mu \mskip1.5mu]}}}{{\uncurry{\Varid{g}}}}}_{[\mskip1.5mu \mskip1.5mu]}\mathbin{\circ}T_{[]}\Varid{f})\mathbin{\circ}\Varid{zip'}\;\mathbf{where}\;\Varid{g}\;\Varid{x}\;\Varid{ys}\mathrel{=}(\lambda \Varid{z}\mathbin{→}\Varid{x}\;\Varid{z}\mathbin{:}\Varid{ys}\;\Varid{z}){}\<[E]%
\\[\blanklineskip]%
\>[B]{}\mbox{\commentbegin  Absorção-cata  \commentend}{}\<[E]%
\\[\blanklineskip]%
\>[B]{}\Varid{zipWithM'}\;\Varid{f}{}\<[17]%
\>[17]{}\mathrel{=}\cata{\either{\const{\const{[\mskip1.5mu \mskip1.5mu]}}}{{\uncurry{\Varid{g}}}}\mathbin{\circ}B_{[]}(\Varid{f},\Varid{id})}_{[\mskip1.5mu \mskip1.5mu]}\mathbin{\circ}\Varid{zip'}\;\mathbf{where}\;\Varid{g}\;\Varid{x}\;\Varid{ys}\mathrel{=}(\lambda \Varid{z}\mathbin{→}\Varid{x}\;\Varid{z}\mathbin{:}\Varid{ys}\;\Varid{z}){}\<[E]%
\\[\blanklineskip]%
\>[B]{}\mbox{\commentbegin  Def-\text{\ttfamily baseList}  \commentend}{}\<[E]%
\\[\blanklineskip]%
\>[B]{}\Varid{zipWithM'}\;\Varid{f}{}\<[17]%
\>[17]{}\mathrel{=}\cata{\either{\const{\const{[\mskip1.5mu \mskip1.5mu]}}}{{\uncurry{\Varid{g}}}}\mathbin{\circ}(\Varid{id}+\Varid{f}\mathbin{×}\Varid{id})}_{[\mskip1.5mu \mskip1.5mu]}\mathbin{\circ}\Varid{zip'}\;\mathbf{where}\;\Varid{g}\;\Varid{x}\;\Varid{ys}\mathrel{=}(\lambda \Varid{z}\mathbin{→}\Varid{x}\;\Varid{z}\mathbin{:}\Varid{ys}\;\Varid{z}){}\<[E]%
\\[\blanklineskip]%
\>[B]{}\mbox{\commentbegin  Absorção-\ensuremath{+}; Natural-const  \commentend}{}\<[E]%
\\[\blanklineskip]%
\>[B]{}\Varid{zipWithM'}\;\Varid{f}{}\<[17]%
\>[17]{}\mathrel{=}\cata{\either{\const{\const{[\mskip1.5mu \mskip1.5mu]}}}{{\uncurry{\Varid{g}}}\mathbin{\circ}(\Varid{f}\mathbin{×}\Varid{id})}}_{[\mskip1.5mu \mskip1.5mu]}\mathbin{\circ}\Varid{zip'}\;\mathbf{where}\;\Varid{g}\;\Varid{x}\;\Varid{ys}{}\<[100]%
\>[100]{}\mathrel{=}(\lambda \Varid{z}\mathbin{→}\Varid{x}\;\Varid{z}\mathbin{:}\Varid{ys}\;\Varid{z}){}\<[E]%
\\[\blanklineskip]%
\>[B]{}\mbox{\commentbegin  \ensuremath{(\mathbin{\circ})\;\Varid{f}\;\Varid{g}\mathrel{=}\lambda \Varid{x}\to \Varid{f}\;(\Varid{g}\;\Varid{x})}  \commentend}{}\<[E]%
\\[\blanklineskip]%
\>[B]{}\Varid{zipWithM'}\;\Varid{f}{}\<[17]%
\>[17]{}\mathrel{=}\cata{\either{\const{\const{[\mskip1.5mu \mskip1.5mu]}}}{\lambda (\Varid{a},\Varid{b})\mathbin{→}{\uncurry{\Varid{g}}}\;((\Varid{f}\mathbin{×}\Varid{id})\;(\Varid{a},\Varid{b}))}}_{[\mskip1.5mu \mskip1.5mu]}\mathbin{\circ}\Varid{zip'}\;\mathbf{where}\;\Varid{g}\;\Varid{x}\;\Varid{ys}\mathrel{=}(\lambda \Varid{z}\mathbin{→}\Varid{x}\;\Varid{z}\mathbin{:}\Varid{ys}\;\Varid{z}){}\<[E]%
\\[\blanklineskip]%
\>[B]{}\mbox{\commentbegin  Def-\ensuremath{\mathbin{×}}  \commentend}{}\<[E]%
\\[\blanklineskip]%
\>[B]{}\Varid{zipWithM'}\;\Varid{f}{}\<[17]%
\>[17]{}\mathrel{=}\cata{\either{\const{\const{[\mskip1.5mu \mskip1.5mu]}}}{\lambda (\Varid{a},\Varid{b})\mathbin{→}\Varid{g}\;(\Varid{f}\;\Varid{a},\Varid{b})}}_{[\mskip1.5mu \mskip1.5mu]}\mathbin{\circ}\Varid{zip'}\;\mathbf{where}\;\Varid{g}\;(\Varid{x},\Varid{ys})\mathrel{=}(\lambda \Varid{z}\mathbin{→}\Varid{x}\;\Varid{z}\mathbin{:}\Varid{ys}\;\Varid{z}){}\<[E]%
\\[\blanklineskip]%
\>[B]{}\mbox{\commentbegin  Deixe que \ensuremath{\Varid{h}\mathrel{=}(\lambda (\Varid{a},\Varid{b})\mathbin{→}\Varid{g}\;(\Varid{f}\;\Varid{a},\Varid{b}))}; Notação-\ensuremath{\lambda }  \commentend}{}\<[E]%
\\[\blanklineskip]%
\>[B]{}\Varid{zipWithM'}\;\Varid{f}{}\<[17]%
\>[17]{}\mathrel{=}\cata{\either{\const{\const{[\mskip1.5mu \mskip1.5mu]}}}{\Varid{h}}}_{[\mskip1.5mu \mskip1.5mu]}\mathbin{\circ}\Varid{zip'}\;\mathbf{where}\;\Varid{h}\;(\Varid{a},\Varid{b})\mathrel{=}(\lambda \Varid{z}\mathbin{→}(\Varid{f}\;\Varid{a})\;\Varid{z}\mathbin{:}\Varid{b}\;\Varid{z}){}\<[E]%
\\[\blanklineskip]%
\>[B]{}\mbox{\commentbegin  Def-\ensuremath{\Varid{zip'}}; Notação-\ensuremath{\lambda }  \commentend}{}\<[E]%
\\[\blanklineskip]%
\>[B]{}\Varid{zipWithM'}\;\Varid{f}{}\<[17]%
\>[17]{}\mathrel{=}\cata{\either{\const{\const{[\mskip1.5mu \mskip1.5mu]}}}{\Varid{h}}}_{[\mskip1.5mu \mskip1.5mu]}\mathbin{\circ}\ana{out_{A^{*}\times B^{*}}}_{[\mskip1.5mu \mskip1.5mu]}\;\mathbf{where}\;\Varid{h}\;(\Varid{a},\Varid{b})\;\Varid{z}\mathrel{=}(\Varid{f}\;\Varid{a})\;\Varid{z}\mathbin{:}\Varid{b}\;\Varid{z}{}\<[E]%
\\[\blanklineskip]%
\>[B]{}\mbox{\commentbegin  catamorfismo após anamorfismo é um hilomorfismo  \commentend}{}\<[E]%
\\[\blanklineskip]%
\>[B]{}\Varid{zipWithM'}\;\Varid{f}{}\<[17]%
\>[17]{}\mathrel{=}\hylo{\either{\const{\const{[\mskip1.5mu \mskip1.5mu]}}}{\Varid{h}},out_{A^{*}\times B^{*}}}_{[\mskip1.5mu \mskip1.5mu]}\;\mathbf{where}\;\Varid{h}\;(\Varid{a},\Varid{b})\;\Varid{z}\mathrel{=}(\Varid{f}\;\Varid{a})\;\Varid{z}\mathbin{:}\Varid{b}\;\Varid{z}{}\<[E]%
\\[\blanklineskip]%
\>[B]{}\Varid{zipWithM'}\;\Varid{f}{}\<[17]%
\>[17]{}\mathrel{=}\hylo{\either{\const{\const{[\mskip1.5mu \mskip1.5mu]}}}{\Varid{h}},out_{A^{*}\times B^{*}}}_{[\mskip1.5mu \mskip1.5mu]}\;\mathbf{where}\;\Varid{h}\;(\Varid{a},\Varid{b})\mathrel{=}\curry{\mathbin{:}}\mathbin{\circ}{⟨\Varid{f}\;\Varid{a},\Varid{b}⟩}{}\<[E]%
\ColumnHook
\end{hscode}\resethooks


Portanto, lembrando que \ensuremath{\Varid{calcLine}\mathrel{=}\Varid{zipWithM}\;\Varid{linear1d}}
e tendo em mente que \ensuremath{\Varid{calcLine}\mathbin{∷}[\mskip1.5mu \BbbQ\mskip1.5mu]\mathbin{→}[\mskip1.5mu \BbbQ\mskip1.5mu]\mathbin{→}\Conid{Float}\mathbin{→}[\mskip1.5mu \BbbQ\mskip1.5mu]},
mas \ensuremath{\Varid{zipWithM'}\mathbin{∷}((\Varid{a}\mathbin{×}\Varid{b})\mathbin{→}\Varid{c}\mathbin{→}\Varid{d})\mathbin{→}([\mskip1.5mu \Varid{a}\mskip1.5mu]\mathbin{×}[\mskip1.5mu \Varid{b}\mskip1.5mu])\mathbin{→}\Varid{c}\mathbin{→}[\mskip1.5mu \Varid{d}\mskip1.5mu]}
\begin{hscode}\SaveRestoreHook
\column{B}{@{}>{\hspre}l<{\hspost}@{}}%
\column{E}{@{}>{\hspre}l<{\hspost}@{}}%
\>[B]{}\Varid{calcLine}\mathrel{=}\curry{\Varid{zipWithM'}\;{\uncurry{\Varid{linear1d}}}}{}\<[E]%
\\[\blanklineskip]%
\>[B]{}\mbox{\commentbegin  Def-\ensuremath{\Varid{zipWithM'15}}  \commentend}{}\<[E]%
\\[\blanklineskip]%
\>[B]{}\Varid{calcLine}\mathrel{=}\curry{\hylo{\either{\const{\const{[\mskip1.5mu \mskip1.5mu]}}}{\Varid{h}},out_{A^{*}\times B^{*}}}_{[\mskip1.5mu \mskip1.5mu]}}\;\mathbf{where}\;\Varid{h}\;(\Varid{a},\Varid{b})\mathrel{=}\curry{\mathbin{:}}\mathbin{\circ}{⟨{\uncurry{\Varid{linear1d}}}\;\Varid{a},\Varid{b}⟩}{}\<[E]%
\ColumnHook
\end{hscode}\resethooks


\begin{hscode}\SaveRestoreHook
\column{B}{@{}>{\hspre}l<{\hspost}@{}}%
\column{3}{@{}>{\hspre}l<{\hspost}@{}}%
\column{4}{@{}>{\hspre}l<{\hspost}@{}}%
\column{5}{@{}>{\hspre}l<{\hspost}@{}}%
\column{8}{@{}>{\hspre}l<{\hspost}@{}}%
\column{19}{@{}>{\hspre}l<{\hspost}@{}}%
\column{32}{@{}>{\hspre}l<{\hspost}@{}}%
\column{E}{@{}>{\hspre}l<{\hspost}@{}}%
\>[B]{}\Varid{deCasteljau}\mathbin{∷}[\mskip1.5mu \Conid{NPoint}\mskip1.5mu]\mathbin{→}\Conid{OverTime}\;\Conid{NPoint}{}\<[E]%
\\
\>[B]{}\Varid{deCasteljau}\mathrel{=}\Varid{hyloAlgForm}\;\Varid{alg}\;{}\<[32]%
\>[32]{}\Varid{coalg}\;\mathbf{where}{}\<[E]%
\\
\>[B]{}\hsindent{4}{}\<[4]%
\>[4]{}\Varid{coalg}\mathrel{=}(\Varid{id}+\Varid{id}+{⟨\Varid{init},\Varid{tail}⟩})\mathbin{\circ}\Varid{outSL}{}\<[E]%
\\
\>[B]{}\hsindent{4}{}\<[4]%
\>[4]{}\Varid{alg}\mathrel{=}\const{\const{[\mskip1.5mu \mskip1.5mu]}}\mathbin{∐}\Varid{a}{}\<[E]%
\\
\>[B]{}\hsindent{4}{}\<[4]%
\>[4]{}\Varid{a}\mathrel{=}\const{{}}\mathbin{∐}\Varid{b}{}\<[E]%
\\
\>[B]{}\hsindent{4}{}\<[4]%
\>[4]{}\Varid{b}\;(\Varid{e},\Varid{d})\;\Varid{pt}\mathrel{=}\Varid{calcLine}\;(\Varid{e}\;\Varid{pt})\;(\Varid{d}\;\Varid{pt})\;\Varid{pt}{}\<[E]%
\\[\blanklineskip]%
\>[B]{}\Varid{outSL}\mathbin{∷}[\mskip1.5mu \Varid{a}\mskip1.5mu]\mathbin{→}()\mathbin{∐}\Varid{a}\mathbin{∐}[\mskip1.5mu \Varid{a}\mskip1.5mu]{}\<[E]%
\\
\>[B]{}\Varid{outSL}\mathrel{=}\lambda \mathbf{case}{}\<[E]%
\\
\>[B]{}\hsindent{3}{}\<[3]%
\>[3]{}[\mskip1.5mu \mskip1.5mu]{}\<[8]%
\>[8]{}\mathbin{→}\Varid{i}_{1}\;(){}\<[E]%
\\
\>[B]{}\hsindent{3}{}\<[3]%
\>[3]{}[\mskip1.5mu \Varid{a}\mskip1.5mu]{}\<[8]%
\>[8]{}\mathbin{→}\Varid{i}_{2}\;(\Varid{i}_{1}\;\Varid{a}){}\<[E]%
\\
\>[B]{}\hsindent{3}{}\<[3]%
\>[3]{}\Varid{l}{}\<[8]%
\>[8]{}\mathbin{→}\Varid{i}_{2}\;(\Varid{i}_{2}\;\Varid{l}){}\<[E]%
\\[\blanklineskip]%
\>[B]{}\Varid{hyloAlgForm}\mathbin{∷}(()\mathbin{∐}\Varid{b}\mathbin{∐}\Varid{c}\mathbin{×}\Varid{c}\mathbin{→}\Varid{c})\mathbin{→}(\Varid{a}\mathbin{→}\Varid{d}\mathbin{∐}\Varid{b}\mathbin{∐}\Varid{a}\mathbin{×}\Varid{a})\mathbin{→}\Varid{a}\mathbin{→}\Varid{c}{}\<[E]%
\\
\>[B]{}\Varid{hyloAlgForm}\mathrel{=}\Varid{h}\;\mathbf{where}{}\<[E]%
\\
\>[B]{}\hsindent{5}{}\<[5]%
\>[5]{}\Varid{h}\;\Varid{a}\;\Varid{b}\mathrel{=}\cata{\Varid{a}}_{Castel}\mathbin{\circ}\ana{\Varid{b}}_{Castel}{}\<[E]%
\\[\blanklineskip]%
\>[B]{}\mathbf{newtype}\;\Conid{Castel'}\;\Varid{a}\mathrel{=}\Conid{Castel'}\;(()\mathbin{∐}\Varid{a}\mathbin{∐}\Conid{Castel}\;\Varid{a}\mathbin{×}\Conid{Castel}\;\Varid{a}){}\<[E]%
\\
\>[B]{}\mathbf{data}\;\Conid{Castel}\;\Varid{a}\mathrel{=}\Conid{Empty}\mid \Conid{Single}\;\Varid{a}\mid \Conid{InitTail}\;(\Conid{Castel}\;\Varid{a}\mathbin{×}\Conid{Castel}\;\Varid{a})\;\mathbf{deriving}\;\Conid{Show}{}\<[E]%
\\[\blanklineskip]%
\>[B]{}in_{\Conid{Castel}}\mathbin{∷}\Varid{b}\mathbin{∐}\Varid{a}\mathbin{∐}\Conid{Castel}\;\Varid{a}\mathbin{×}\Conid{Castel}\;\Varid{a}\mathbin{→}\Conid{Castel}\;\Varid{a}{}\<[E]%
\\
\>[B]{}in_{\Conid{Castel}}\mathrel{=}\const{\Conid{Empty}}\mathbin{∐}\Conid{Single}\mathbin{∐}\Conid{InitTail}{}\<[E]%
\\[\blanklineskip]%
\>[B]{}out_{\Conid{Castel}}\mathbin{∷}\Conid{Castel}\;\Varid{a}\mathbin{→}()\mathbin{∐}\Varid{a}\mathbin{∐}\Conid{Castel}\;\Varid{a}\mathbin{×}\Conid{Castel}\;\Varid{a}{}\<[E]%
\\
\>[B]{}out_{\Conid{Castel}}\mathrel{=}\lambda \mathbf{case}{}\<[E]%
\\
\>[B]{}\hsindent{3}{}\<[3]%
\>[3]{}\Conid{Empty}{}\<[19]%
\>[19]{}\mathbin{→}\Varid{i}_{1}\;(){}\<[E]%
\\
\>[B]{}\hsindent{3}{}\<[3]%
\>[3]{}\Conid{Single}\;\Varid{a}{}\<[19]%
\>[19]{}\mathbin{→}\Varid{i}_{2}\;(\Varid{i}_{1}\;\Varid{a}){}\<[E]%
\\
\>[B]{}\hsindent{3}{}\<[3]%
\>[3]{}\Conid{InitTail}\;(\Varid{e},\Varid{d}){}\<[19]%
\>[19]{}\mathbin{→}\Varid{i}_{2}\;(\Varid{i}_{2}\;(\Varid{e},\Varid{d})){}\<[E]%
\\[\blanklineskip]%
\>[B]{}T_{Castel}\mathbin{∷}(\Varid{a}\mathbin{→}\Varid{d})\mathbin{→}\Varid{b1}\mathbin{∐}\Varid{b2}\mathbin{∐}\Varid{a}\mathbin{×}\Varid{a}\mathbin{→}\Varid{b1}\mathbin{∐}\Varid{b2}\mathbin{∐}\Varid{d}\mathbin{×}\Varid{d}{}\<[E]%
\\
\>[B]{}T_{Castel}\;\Varid{f}\mathrel{=}\Varid{id}+\Varid{id}+\Varid{f}\mathbin{×}\Varid{f}{}\<[E]%
\\[\blanklineskip]%
\>[B]{}\cata{{}}_{Castel}\mathbin{∷}(()\mathbin{∐}\Varid{b}\mathbin{∐}\Varid{d}\mathbin{×}\Varid{d}\mathbin{→}\Varid{d})\mathbin{→}\Conid{Castel}\;\Varid{b}\mathbin{→}\Varid{d}{}\<[E]%
\\
\>[B]{}\cata{\Varid{f}}_{Castel}\mathrel{=}\Varid{f}\mathbin{\circ}T_{Castel}\;\cata{\Varid{f}}_{Castel}\mathbin{\circ}out_{\Conid{Castel}}{}\<[E]%
\\
\>[B]{}\ana{{}}_{Castel}\mathbin{∷}(\Varid{a1}\mathbin{→}\Varid{b}\mathbin{∐}\Varid{a2}\mathbin{∐}\Varid{a1}\mathbin{×}\Varid{a1})\mathbin{→}\Varid{a1}\mathbin{→}\Conid{Castel}\;\Varid{a2}{}\<[E]%
\\
\>[B]{}\ana{\Varid{g}}_{Castel}\mathrel{=}in_{\Conid{Castel}}\mathbin{\circ}T_{Castel}\;\ana{\Varid{g}}_{Castel}\mathbin{\circ}\Varid{g}{}\<[E]%
\ColumnHook
\end{hscode}\resethooks

\subsection{Problema 4}
\begin{hscode}\SaveRestoreHook
\column{B}{@{}>{\hspre}l<{\hspost}@{}}%
\column{3}{@{}>{\hspre}l<{\hspost}@{}}%
\column{9}{@{}>{\hspre}l<{\hspost}@{}}%
\column{E}{@{}>{\hspre}l<{\hspost}@{}}%
\>[B]{}out_{\Conid{NList}}\mathbin{∷}[\mskip1.5mu \Varid{a}\mskip1.5mu]\mathbin{→}\Varid{a}\mathbin{∐}\Varid{a}\mathbin{×}[\mskip1.5mu \Varid{a}\mskip1.5mu]{}\<[E]%
\\
\>[B]{}out_{\Conid{NList}}\mathrel{=}\lambda \mathbf{case}{}\<[E]%
\\
\>[B]{}\hsindent{3}{}\<[3]%
\>[3]{}[\mskip1.5mu \Varid{a}\mskip1.5mu]{}\<[9]%
\>[9]{}\mathbin{→}\Varid{i}_{1}\;\Varid{a}{}\<[E]%
\\
\>[B]{}\hsindent{3}{}\<[3]%
\>[3]{}(\Varid{a}\mathbin{:}\Varid{x})\mathbin{→}\Varid{i}_{2}\;(\Varid{a},\Varid{x}){}\<[E]%
\ColumnHook
\end{hscode}\resethooks

\begin{hscode}\SaveRestoreHook
\column{B}{@{}>{\hspre}l<{\hspost}@{}}%
\column{7}{@{}>{\hspre}l<{\hspost}@{}}%
\column{11}{@{}>{\hspre}l<{\hspost}@{}}%
\column{E}{@{}>{\hspre}l<{\hspost}@{}}%
\>[B]{}rec_{\Conid{NList}}\mathbin{∷}(\Varid{c}\mathbin{→}\Varid{d})\mathbin{→}(\Varid{b1}\mathbin{∐}\Varid{b2}\mathbin{×}\Varid{c})\mathbin{→}\Varid{b1}\mathbin{∐}\Varid{b2}\mathbin{×}\Varid{d}{}\<[E]%
\\
\>[B]{}rec_{\Conid{NList}}\;{}\<[7]%
\>[7]{}\Varid{f}{}\<[11]%
\>[11]{}\mathrel{=}\Varid{id}+\Varid{id}\mathbin{×}\Varid{f}{}\<[E]%
\\[\blanklineskip]%
\>[B]{}\cata{{}}_{\Conid{NList}}\mathbin{∷}(\Varid{b}\mathbin{∐}\Varid{b}\mathbin{×}\Varid{d}\mathbin{→}\Varid{d})\mathbin{→}[\mskip1.5mu \Varid{b}\mskip1.5mu]\mathbin{→}\Varid{d}{}\<[E]%
\\
\>[B]{}\cata{\Varid{g}}_{\Conid{NList}}{}\<[11]%
\>[11]{}\mathrel{=}\Varid{g}\mathbin{\circ}rec_{\Conid{NList}}\;\cata{\Varid{g}}_{\Conid{NList}}\mathbin{\circ}out_{\Conid{NList}}{}\<[E]%
\ColumnHook
\end{hscode}\resethooks

\begin{hscode}\SaveRestoreHook
\column{B}{@{}>{\hspre}l<{\hspost}@{}}%
\column{4}{@{}>{\hspre}l<{\hspost}@{}}%
\column{7}{@{}>{\hspre}l<{\hspost}@{}}%
\column{E}{@{}>{\hspre}l<{\hspost}@{}}%
\>[B]{}\Varid{avg\char95 aux}\mathbin{∷}\Conid{Fractional}\;\Varid{b}\mathbin{⇒}[\mskip1.5mu \Varid{b}\mskip1.5mu]\mathbin{→}(\Varid{b}\mathbin{×}\Varid{b}){}\<[E]%
\\
\>[B]{}\Varid{avg\char95 aux}\mathrel{=}\cata{\either{\Varid{b}}{\Varid{q}}}_{\Conid{NList}}\;\mathbf{where}{}\<[E]%
\\
\>[B]{}\hsindent{4}{}\<[4]%
\>[4]{}\Varid{b}\;\Varid{a}\mathrel{=}(\Varid{a},\mathrm{1}){}\<[E]%
\\
\>[B]{}\hsindent{4}{}\<[4]%
\>[4]{}\Varid{q}\;(\Varid{h},(\Varid{a},\Varid{l}))\mathrel{=}((\Varid{h}\mathbin{+}(\Varid{a}\mathbin{*}\Varid{l}))\mathbin{/}(\Varid{l}\mathbin{+}\mathrm{1}),\Varid{l}\mathbin{+}\mathrm{1}){}\<[E]%
\\[\blanklineskip]%
\>[B]{}\Varid{avgLTree}\mathbin{∷}\Conid{Fractional}\;\Varid{b}\mathbin{⇒}\Conid{LTree}\;\Varid{b}\mathbin{→}\Varid{b}{}\<[E]%
\\
\>[B]{}\Varid{avgLTree}\mathrel{=}\Varid{p}_{1}\mathbin{\circ}\Varid{cataLTree}\;\Varid{gene}\;\mathbf{where}{}\<[E]%
\\
\>[B]{}\hsindent{4}{}\<[4]%
\>[4]{}\Varid{gene}\mathrel{=}\either{\Varid{g}}{\Varid{q}}\;\mathbf{where}{}\<[E]%
\\
\>[4]{}\hsindent{3}{}\<[7]%
\>[7]{}\Varid{g}\;\Varid{a}\mathrel{=}(\Varid{a},\mathrm{1}){}\<[E]%
\\
\>[4]{}\hsindent{3}{}\<[7]%
\>[7]{}\Varid{q}\;((\Varid{a}_{1},\Varid{l}_{1}),(\Varid{a}_{2},\Varid{l}_{2}))\mathrel{=}(((\Varid{a}_{1}\mathbin{*}\Varid{l}_{1})\mathbin{+}(\Varid{a}_{2}\mathbin{*}\Varid{l}_{2}))\mathbin{/}(\Varid{l}_{1}\mathbin{+}\Varid{l}_{2}),\Varid{l}_{1}\mathbin{+}\Varid{l}_{2}){}\<[E]%
\ColumnHook
\end{hscode}\resethooks




\end{document}
